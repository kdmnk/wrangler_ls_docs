\chapter{User documentation}
\label{ch:user}

The Wrangler Language Server is a tool that enables developers to use the refactorings defined in Wrangler through the Erlang Language Server.

In this paper, the set up, the examples and the screenshots are presented for the development tool called Visual Studio Code owned my Microsoft. \todo{recompose}

Installing Wrangler includes installing the Wrangler Language Server, but it is started by and ran in the Erlang Language Server (that is started by and ran in the development tool). \todo{check phrasing}
In order to use the Wrangler Language Server, further configuration is needed in the Erlang Language Server.

\section{System requirements}

The system requirements for the Wrangler Language Server is constrained by the underlying tools' (namely the Erlang Language Server, Wrangler itself and the development tool: Visual Studio Code) requirements:

\begin{enumerate}
    \item Hardware requirements: \todo{Check harware requirements}
    \begin{itemize}
        \item 1.6 GHz or faster processor is recommended by Visual Studio Code 
        \cite{VSCodeRequirements}
        \item 1 GB of RAM is recommended by Visual Studio Code  \cite{VSCodeRequirements}
        \item 500\todo{+x} MB free space on the hard drive is needed to install Visual Studio Code, Wrangler and the Erlang
    \end{itemize}
    \item Software requirements:
    \begin{itemize}
        \item GNU/Linux or MacOS operating system \footnote{The Microsoft Windows operating system is restricted only by Wrangler`s currently outdated Windows installer.}
        \item Erlang/OTP 24.0 or newer version.
        \item \todo{Wrangler requirements - gcc?}
    \end{itemize}
    \item Other requirements:
    \begin{itemize}
        \item Administrator privileges on the operating system.
    \end{itemize}
\end{enumerate}


\section{Installing and configuration}

\subsection{Installing Wrangler}

First, download or clone the Wrangler code base from its GitHub repository \cite{WranglerGitHub}. More information about cloning repositories can be found in GitHub Docs \cite{GitCloneTutorial}.

Then build and install Wrangler with the command below \ref{src:install}.

\lstset{caption={Build and install Wrangler}, label=src:install}
\begin{lstlisting}[language=bash]
  ./configure && make && sudo make install
\end{lstlisting}

More information and options about installing Wrangler can be found in the \emph{INSTALL} file located at Wrangler`s root directory.

As mentioned, installing Wrangler will also install the Wrangler Language Server.

\subsection{Installing Visual Studio Code}

A detailed walk-through is available about installing Visual Studio Code at their website \cite{VSCodeDownload}.

\subsection{Installing the Erlang Language Server}

The Erlang Language Server for Visual Studio Code is available as a dedicated extension called \emph{Erlang LS} \cite{ELSExtension}.

A guide to install the Erlang Language Server in Visual Studio Code (and also for other development tools) is available at their website \cite{ELSInstall}.

\subsection{Configuring the Wrangler Language Server}

A configuration file named \emph{erlang\_ls.config} should be placed in the root directory of every project where Wrangler need to be used.

A configuration example \ref{src:wrangler-config-example} and the description of the parameters \ref{tab:wrangler-config-descr} are shown below.

\lstset{caption={erlang\_ls.config example}, label=src:wrangler-config-example}
\begin{lstlisting}
wrangler:
  enabled: true
  path: "/path/to/wrangler/ebin" 
  tab_with: 8
  enabled_refactorings:
    - "comment-out-spec"
    - "fold-expression"
    - "generalise-fun"
    - "move-fun"
    - "new-fun"
    - "new-macro"
    - "new-var"
    - "rename-fun2"
    - "rename-var"
\end{lstlisting}


\begin{table}[H]
	\centering
	\begin{tabular}{ | m{0.3\textwidth} | m{0.1\textwidth} | m{0.5\textwidth} | }
		\hline
		\textbf{Parameter} & \textbf{Type} & \textbf{Description} \\
		\hline \hline
		\emph{enabled} & \emph{boolean} & \emph{true} if Wrangler should be enabled or \emph{false} if not. \\
		\hline
		\emph{path} & \emph{string?} & Only necessary when Wrangler is not installed system-wide. A path to Wrangler's compiled \emph{ebin} folder should be set. \\
		\hline
		\emph{tab\_with} & \emph{integer?} & The used tabulator width in the project. Defaults to 8 if not provided. \\
		\hline
		\emph{enabled\_refactorings} & \emph{string[]} & All the Wrangler refactoring`s identification values that are enabled.\\
		\hline
	\end{tabular}
	\caption{Description of the configuration parameters}
	\label{tab:wrangler-config-descr}
\end{table}

Further configuration options for the Erlang Language Server can be found at their website \cite{ELSConfig}.

\section{Usage}

The Erlang Language Server starts automatically when an Erlang file is opened in the code editor. If Wrangler is enabled, the Erlang Language Server loads and starts it during initialization. After that, the refactorings should be accessible.

\subsection{General case}
\subsection{Refactorings}

\subsubsection{Rename Variable}
\subsubsection{...}


